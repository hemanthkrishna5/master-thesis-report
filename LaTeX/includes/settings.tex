%%%%%%%%%%%%%%%%%%%%%%%%%%%%%%%%%%%%%%%%%%%%%%%%%%%%%%%%%%%%%%%%%%%%%%%
%% Global Document Settings
%%%%%%%%%%%%%%%%%%%%%%%%%%%%%%%%%%%%%%%%%%%%%%%%%%%%%%%%%%%%%%%%%%%%%%


%%%%%%%%%%%%%%%%%%%%%%%%%%%%%%%%%%%%%%%%%%%%%%%%%%%%%%%%%%%%%%%%%%%%%%%
%% Headers & Footers
%%%%%%%%%%%%%%%%%%%%%%%%%%%%%%%%%%%%%%%%%%%%%%%%%%%%%%%%%%%%%%%%%%%%%%

% page style for front page
\fancypagestyle{frontpage}{
	\fancyhf{}
	
	%remove top line
	\renewcommand{\headrulewidth}{0pt}

	\fancyfoot[L]{
		\color{black} \footnotesize
	}
	\fancyfoot[C]{
		\color{black} \footnotesize
	}
	\fancyfoot[R]{
		\color{black} \footnotesize
		\iftoggle{isGerman}{
  			Seite \thepage
		}{
			Page \thepage
		}
	}
}

% page style for all other pages
\fancypagestyle{otherpages}{
	
	\fancyhf{}

	%remove top line
	%\renewcommand{\headrulewidth}{0pt}

	\fancyhead[L]{\nouppercase{\leftmark}
		\color{black} \footnotesize
	}
	\fancyhead[C]{
		\color{black} \footnotesize
	}
	\fancyhead[R]{
		\color{black} \footnotesize
	}

	\fancyfoot[L]{
		\color{black} \footnotesize
	}
	\fancyfoot[C]{
		\color{black} \footnotesize
	}
	\fancyfoot[R]{
		\color{black} \footnotesize
		\iftoggle{isGerman}{
  			Seite \thepage
		}{
			Page \thepage
		}
	}
}

\fancypagestyle{firstsection}{%
	\fancyhf{}%
	
	\fancyhead[L]{
		\color{black} \footnotesize
	}
	\fancyhead[C]{
		\color{black} \footnotesize
	}
	\fancyhead[R]{
		\color{black} \footnotesize
	}
	
	\fancyfoot[L]{
		\color{black} \footnotesize
	}
	\fancyfoot[C]{
		\color{black} \footnotesize
	}
	\fancyfoot[R]{
		\color{black} \footnotesize
		\iftoggle{isGerman}{
			Seite \thepage
		}{
			Page \thepage
		}
	}
}


% standard page style except for frontpage
\pagestyle{otherpages}



%%%%%%%%%%%%%%%%%%%%%%%%%%%%%%%%%%%%%%%%%%%%%%%%%%%%%%%%%%%%%%%%%%%%%%%
%% Glossaries
%%%%%%%%%%%%%%%%%%%%%%%%%%%%%%%%%%%%%%%%%%%%%%%%%%%%%%%%%%%%%%%%%%%%%%
\usepackage{siunitx}
\usepackage[acronym,symbols,nogroupskip]{glossaries-extra}

%\setlength{\glsdescwidth}{.72\textwidth}

% new keys must be defined before use
\glsaddstoragekey{unit}{}{\glsentryunit}
\glsnoexpandfields

% Style für Glossar
\newglossaryentry{anthropogen}
{
	type=main,
	name=anthropogen,
	description={Der Begriff bezeichnet etwas, was durch den Menschen verursacht oder beeinflusst wird. Im Kontext des Klimawandels beschreibt der Begriff den Einfluss des Menschen auf das Klima der Erde.}
} % Glossareinträge hier einfügen

\newglossarystyle{glossar}{%
	\setglossarystyle{long3col}% base this style on the list style
	\renewenvironment{theglossary}{% Change the table type --> 3 columns
		\begin{longtable}{|p{.17\textwidth}|p{.774\textwidth}|}}%
		{\end{longtable}}%
	%
	\renewcommand*{\glossaryheader}{%  Change the table header
		\hline \bfseries Begriff & \bfseries Beschreibung  \\\hline
		\endhead}%
	\renewcommand*{\glossentry}[2]{%  Change the displayed items
		\glstarget{##1}{\glossentryname{##1}} %
		& \glossentrydesc{##1}% Description
		\tabularnewline \hline
	}%
}

% Style für Abkürzungsverzeichnis
% verwendete Abkürzungen

\newacronym{bmwi}{BMWI}{Bundesministerium für Wirtschaft und Energie}
\newacronym{ch4}{\ensuremath{\text{CH}_{\text{4}}}}{Methan}
\newacronym{co2}{\ensuremath{\text{CO}_{\text{2}}}}{Kohlenstoffdioxid}
\newacronym{ml}{ML}{Maschinelles Lernen}
\newacronym{nh3}{\ensuremath{\text{NH}_{\text{3}}}}{Ammoniak}
\newacronym{thg}{THG}{Treibhausgas}

%%%%%%%%%%%%%%%%%%%%%%%%%%%%%%%%%%%%%%%%%%%%%%%%%%%%%%%%%%%%%%%%%%%%%%%%%%%%%%
% example
%%%%%%%%%%%%%%%%%%%%%%%%%%%%%%%%%%%%%%%%%%%%%%%%%%%%%%%%%%%%%%%%%%%%%%%%%%%%%%

% \newacronym{utc}{UTC}{Coordinated Universal Time}
% \newacronym{adt}{ADT}{Atlantic Daylight Time}
% \newacronym{est}{EST}{Eastern Standard Time}

% Reference acronyms: \gls{UTC}

% Don’t use \gls in chapter or section headings as it can have some unpleasant side-effects. Instead use \glsentrytext for regular entries and one of \glsentryshort, \glsentryshortpl, \glsentrylong, \glsentrylongpl or \glsentryfull for acronyms. Alternatively use glossaries-extra which provides special commands for use in section headings and captions, such as \glsfmtshort{<label>}.
% http://tug.ctan.org/macros/latex/contrib/glossaries/glossariesbegin.pdf % Abkürzungen hier einfügen

\NewEnviron{acronymtable}{%
    \begin{xltabular}{\textwidth}{|c|X|}%
        \BODY
    \end{xltabular}
}
\newglossarystyle{acronyms}{%
	\setglossarystyle{long3col}% base this style on the list style
	\renewenvironment{theglossary}{% Change the table type --> 3 columns
    \acronymtable
    }{
    \endacronymtable
    }
	%
	\renewcommand*{\glossaryheader}{%  Change the table header
		\hline \bfseries Abkürzung & \bfseries Bedeutung  \\\hline
		\endhead}%
	\renewcommand*{\glossentry}[2]{%  Change the displayed items
		\glstarget{##1}{\glossentryname{##1}} %
		& \glossentrydesc{##1}% Description
		  \tabularnewline \hline
	}%
}

% Style für Symbolverzeichnis
% verwendete Symbole

% example
\glsxtrnewsymbol[description={Strom},unit={\si{$\mathrm{\ampere}$}}]{I}{\ensuremath{I}}
\glsxtrnewsymbol[description={Spannung},unit={\si{$\mathrm{\volt}$}}]{U}{\ensuremath{U}} % Symbole hier einfügen

\NewEnviron{symboltable}{%
    \begin{xltabular}{\textwidth}{|c|X|c|}%
        \BODY
    \end{xltabular}
}
\newglossarystyle{symbunitlong}{%
	\setglossarystyle{long3col}% base this style on the list style
	\renewenvironment{theglossary}{%
    \symboltable
    }{
    \endsymboltable
    }
	\renewcommand*{\glossaryheader}{%  Change the table header
		\hline \bfseries Variable & \bfseries Bedeutung & \bfseries Basiseinheit\\\hline
		\endhead}%
	\renewcommand*{\glossentry}[2]{%  Change the displayed items
		\glstarget{##1}{\glossentryname{##1}} %
		& \glossentrydesc{##1}% Description
		& \glsentryunit{##1}  \tabularnewline \hline
	}%
}

%% glossaries package for list of abbreviations and list of symbols
%\usepackage{datatool} % package needed by glossaries
%\usepackage[acronym]{glossaries} % *after* hyperref
%\newglossary[symlog]{symbol}{symi}{symo}{Symbols}
%\makeglossaries
%\glstoctrue
%\input{../../paper.library/abbreviations_symbols/abbreviations}
%\input{../../paper.library/abbreviations_symbols/symbols_finance}