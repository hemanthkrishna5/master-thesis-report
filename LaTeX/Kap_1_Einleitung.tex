\section{Einleitung} \label{Einleitung} \thispagestyle{firstsection}
Diese Vorlage entspricht den neuen Regeln des Brand Designs der RPTU.
Hierzu wurde die Vorlage von \url{https://github.com/RPTU-EIT/report-eit-latex} modifiziert.
Im Unterordner \texttt{style-eit-latex} ist die Datei \texttt{EIT.sty} hinterlegt.
In dieser Style-Datei wurden die Farben und die Schriftart auf das RPTU-Design umgestellt.
Aufgrund der notwendigen Verwendung bestimmter Packages wird diese Vorlage nur in LuaLaTex unterstützt!\\
Folgende Schriftarten wurden als Standard integriert:
\begin{table}[ht]
    \label{tab:Standardschriftart}
    \caption{RedHatText-Standardschriftart und ihre Formen}
    \centering
    \footnotesize
    \begin{tabular}{@{}ccc@{}}
    \toprule
    Form            & Name der Schriftart                           & Befehl\\ \midrule
    normal          & RedHatText-Regular                            & - \\
    fett            & \textbf{RedHatText-SemiBold}                  & \verb|\bfseries| oder \verb|\textbf{}|\\
    kursiv          & \textit{RedHatText-Italic}                    & \verb|\itshape| oder \verb|\textit{}|\\
    fett \& kursiv  & \textbf{\textit{RedHatText-SemiBoldItalic}}   & \verb|\bfseries\itshape| oder \verb|\textbf{\textit{}}|\\ \bottomrule
    \end{tabular}
\end{table}

Des Weiteren wurden die anderen verfügbaren RedHatText-Schriftarten integriert:
\begin{table}[ht]
    \label{tab:restliche_Schriftarten}
    \caption{Weitere RedHatText-Schriftarten und ihre Formen}
    \centering
    \footnotesize
    \begin{tabular}{@{}ccc@{}}
    \toprule
    Form                    & Name der Schriftart& Befehl\\ \midrule
    light                   & \fontseries{li}\fontshape{n}\selectfont RedHatText-Light          & \verb|\fontseries{li}\fontshape{n}\selectfont|    \\
    light \& kursiv         & \fontseries{li}\fontshape{it}\selectfont RedHatText-LightItalic   & \verb|\fontseries{li}\fontshape{it}\selectfont|   \\
    medium                  & \fontseries{md}\fontshape{n}\selectfont RedHatText-Medium         & \verb|\fontseries{md}\fontshape{n}\selectfont|    \\
    medium \& kursiv        & \fontseries{md}\fontshape{it}\selectfont RedHatText-MediumItalic  & \verb|\fontseries{md}\fontshape{it}\selectfont|   \\
    extra fett              & \fontseries{bf}\fontshape{n}\selectfont RedHatText-Bold           & \verb|\fontseries{bf}\fontshape{n}\selectfont|    \\
    extra fett \& kursiv    & \fontseries{bf}\fontshape{it}\selectfont RedHatText-BoldItalic    & \verb|\fontseries{bf}\fontshape{it}\selectfont|   \\ \bottomrule
    \end{tabular}
\end{table}

\tcbset{
    frame code={}
    center title,
    left=0pt,
    right=0pt,
    top=0pt,
    bottom=0pt,
    colback=gray!70,
    colframe=white,
    width=30mm,
    enlarge left by=0mm,
    boxsep=0pt,
    arc=0pt,outer arc=0pt,
    }

\clearpage
Die Schriftfarbe kann über den \verb|\textcolor{<Farbe>}{<Text>}| Befehl eingestellt werden.
Folgende Farben stehen zur Verfügung:

\begin{table}[ht]
    \label{tab:Farben}
    \caption{RPTU-Farben}
    \centering
    \footnotesize
    \begin{tabular}{@{}cc@{}}
    \toprule
    Farbe & Befehl\\ \midrule
	\bfseries\textcolor{rptublaugrau}{blaugrau}                           & \verb|\textcolor{rptublaugrau}{blaugrau}|       \\
	\bfseries\textcolor{rptugruengrau}{gruengrau}                         & \verb|\textcolor{rptugruengrau}{gruengrau}|     \\
	\bfseries\textcolor{rptudunkelblau}{dunkelblau}                       & \verb|\textcolor{rptudunkelblau}{dunkelblau}|   \\
	\bfseries\textcolor{rptuhellblau}{hellblau}                           & \verb|\textcolor{rptuhellblau}{hellblau}|       \\
	\bfseries\textcolor{rptudunkelgruen}{dunkelgruen}                     & \verb|\textcolor{rptudunkelgruen}{dunkelgruen}| \\
	\bfseries\textcolor{rptuhellgruen}{hellgruen}                         & \verb|\textcolor{rptuhellgruen}{hellgruen}|     \\
	\bfseries\textcolor{rptuviolett}{violett}                             & \verb|\textcolor{rptuviolett}{violett}|         \\
	\bfseries\textcolor{rptupink}{pink}                                   & \verb|\textcolor{rptupink}{pink}|               \\
	\bfseries\textcolor{rpturot}{rot}                                     & \verb|\textcolor{rpturot}{rot}|                 \\
	\bfseries\textcolor{rptuorange}{orange}                               & \verb|\textcolor{rptuorange}{orange}|           \\
	\bfseries\textcolor{rptuschwarz}{schwarz}                             & \verb|\textcolor{rptuschwarz}{schwarz}|         \\
	\bfseries\begin{tcolorbox}\centering\textcolor{rptuweiss}{weiss}\end{tcolorbox} & \verb|\textcolor{rptuweiss}{weiss}|   \\ \bottomrule
    \end{tabular}
\end{table}

\clearpage